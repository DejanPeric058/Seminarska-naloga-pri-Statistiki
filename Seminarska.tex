\documentclass[12pt, a4paper]{article}
\usepackage[slovene,english]{babel}
\usepackage[utf8]{inputenc}
\usepackage{amsmath, amsfonts, amsthm, url, amssymb}
\usepackage{lmodern}
\usepackage{graphicx}
\usepackage[shortlabels]{enumitem}
\setlength{\parindent}{0cm}

\newtheorem{definicija*}{Definicija}
\newtheorem{trditev}[definicija*]{Trditev}
\newtheorem{posledica}[definicija*]{Posledica}
\newtheorem{izrek}[definicija*]{Izrek}
\newtheorem*{primer}{Primer}


\title{Seminarska naloga iz Statistike}
\author{Dejan Perić \\~ \\ }


\date{\today}

%%%%%%%%%%%%%%%%%%%%%%%%%%%%%%%%%%%%%%%%%%%%%%%%%%%%%%%%%%%%%%%%%%%%%%%%%%


\begin{document}
\selectlanguage{slovene}

\maketitle



%%%%%%%%%%%%%%%%%%%%%%%%%%%%%%%%%%%%%%%%%%%%%%%%%%%%%%%%%%%%%%%%%%%%%%%%%%

\section{Prva naloga}

V datoteki Kibergrad se nahajajo informacije o 43.886 družinah, 
ki stanujejo v mestu Kibergrad. Za vsako družino so zabeleženi 
naslednji podatki (ne boste potrebovali vseh):

\begin{itemize}
    \item Tip družine (od 1 do 3)
    \item Število članov družine
    \item Število otrok v družini
    \item Skupni dohodek družine
    \item Mestna četrt, v kateri stanuje družina (od 1 do 4)
    \item Stopnja izobrazbe vodje gospodinjstva (od 31 do 46)
\end{itemize}
    
\subsection{a)}

Izmed 43886 družin izberemo vzorec družin velikosti 200. Vzorce bomo pridobivali s pomočjo sample iz knjižnice random. Za cenilko povprečnega števila otrok bomo vzeli cenilko

$$\hat{\mu} = \frac{1}{200}\sum^{200}_{i=1} X_i $$

Dobimo oceno, da je povprečno število otrok v Kibergradu enako $0,9400$.

\subsection{b)}

Za cenilko standardne napake bomo vzeli cenilko iz predavanj. Ta je enaka 

$$ \hat{SE}_+^2 = \frac{N-n}{N-1} \cdot \frac{\hat{\sigma}_+^2}{n} \text{ ,}$$

pri čemer je $\hat{\sigma}_+^2$ enaka 

$$\hat{\sigma}_+^2 = \frac{N-1}{N} \cdot \frac{1}{n-1} \sum_{i=1}^{n} (X_i - X)^2 \text{ .} $$ $$
\hat{SE}_+^2 = \frac{N-n}{N} \cdot \frac{1}{n\cdot(n-1)} \sum_{i=1}^{n} (X_i - X)^2
$$

Pri nas je $ N = 43886 $ in $ n = 200 $, torej je

$$ \hat{SE}_+^2 = \frac{43686}{43886} \cdot \frac{1}{200\cdot199} \sum_{i=1}^{200} (X_i - X)^2 \text{ .}
$$

Izračunamo, da je standardna napaka enaka $ 0.07800452720101289 $.

Sedaj s pomočjo studentove porazdelitve izračunajmo eksakten 95\% interval zaupanja.
Iz predavanj vemo, da je ta oblike

$$ \mu \in (\hat{\mu} - \hat{SE}_+ \cdot F^{-1}_{Student(n-1)}(1-\frac{\alpha}{2}),\hat{\mu} + \hat{SE}_+ \cdot F^{-1}_{Student(n-1)}(1-\frac{\alpha}{2}) ) \text{ .}
$$

Oceni za povprečno število otrok ($\hat{\mu}$) in standardno napako ($\hat{SE}_+$)
že imamo. Stopnja tveganja je enaka $\alpha = 0,05$. S pomočjo knjižnice $scipy.stats$ izračunamo, da je 

$$ F^{-1}_{Student(n-1)}(0,975) = 1.9719565442493954
$$

Interval zaupanja je torej enak $(0.7861784621048826, 1.0938215378951173)$

\subsection{c)}

Za cenilko povprečja celotne populacije bomo vzeli isto kot smo za oceno populacijskega
povprečja pri vzorcu. 

$$ \mu = \frac{1}{43886}\sum^{200}_{i=1} X_i = 0.9479332816843641
$$

Opazimo, da se populacijsko povprečje in vzorčno povprečje ujemata zgolj v eni decimalki.
Iz vseh podatkov družin lahko izračunamo varianco števila otrok družin, s pomočjo
katere bomo lahko ocenili kakšna je prava standardna napaka za vzorec velikosti 200.
Ker imamo podatke celotne populacije, za cenilko variance vzamemo

$$ \sigma^2 = \frac{1}{n} \sum_{i=1}^{43886} (X_i - X)^2 = 1.3391064929282408
$$

Za cenilko prave standardne napake vzamemo cenilko, najdeno v knjigi ????:

$$ SE^2 = \frac{\sigma^2}{n} \cdot (\frac{N-n}{N-1}) = \frac{\sigma^2}{200} \cdot \frac{43686}{43885}
$$

Dobimo, da je prava standardna napaka ocenjena z $SE = 0.0816404987959038$. Absolutna
napaka ocenjene standardne napake je torej ???, relativna pa ???.

Opazimo, da je

$$ \mu = 0.9479332816843641 \in (0.7379511969254118, 1.0620488030745883) \text{ ,}
$$

torej interval zaupanja iz prejšnje točke pokrije populacijsko povprečje.

\subsection{d)}

Vzemimo še 99 vzorcev populacije velikosti 200. Določimo 95\% intervale zaupanja. 
Glede na to, da imamo skupaj 100
vzorcev, pričakujemo, da bo približno 95 intervalov zaupanja pokrilo populacijsko
povprečje ($\mu = ???$). Vse skupaj ponazorimo z grafom.

% include graphics

Ugotovimo, da ??? intervalov pokrije populacijsko povprečje.

\subsection{e)}

Za teh 99 vzorcev izračunamo še standardni odklon za vsakega posebej. 
Za cenilko standardnega odklona vzamemo isto, kot smo jo vzeli za standardno 
napako v prvem vzorcu; standardni odklon in standardna napaka sta namreč za ta
primer ista. Skupaj s pravo standardno napako rezultate prikažemo z grafom.

% include graphics



\subsection{f)}
Izvedite prejšnji dve točki še na 100 vzorcih po 800 družin. Primerjajte 
in razložite razlike s teorijo vzorčenja.

Sedaj vzamemo 100 vzorcev populacije velikosti 400. Za vsak vzorec izračunamo
interval zaupanja in standardni odklon. Ker imamo 95\% interval zaupanja, spet
pričakujemo, da bo približno 95 intervalov pokrilo populacijsko povprečje.

% include graphics

Vidimo, da ??? pokrije interval.

Poglejmo si še graf z ocenjenimi standardnimi odkloni.

% include graphics

Opazimo, da je odstopanje od pravega standardnega odklona sedaj manjše kot
pri vzorcih velikosti 200.


%%%%%%%%%%%%%%%%%%%%%%%%%%%%%%%%%%%%%%%%%%%%%%%%%%%%%%%%%%%%%%%%%%%%%%%%%%

\section{Druga naloga}

\subsection{a)}
Ocenite povprečje in standardni odklon za telesno temperaturo posebej pri
moških in posebej pri ženskah.

\subsection{b)}
Za povprečji iz prejšnje točke določite 95\% intervala zaupanja.

\subsection{c)}
Preizkusite domnevo, da imajo moški in ženske v povprečju enako telesno 
temperaturo.


%%%%%%%%%%%%%%%%%%%%%%%%%%%%%%%%%%%%%%%%%%%%%%%%%%%%%%%%%%%%%%%%%%%%%%%%%%

\section{Tretja naloga}

V datoteki Temp LJ se nahajajo izmerjene mesečne temperature v Ljubljani v 
letih od 1986 do 2020. Postavimo naslednja dva modela spreminjanja temperature
s časom: 

\begin{itemize}
     
    \item Model A: vključuje linearni trend in sinusno nihanje s periodo eno 
        leto.
    \item Model B: vključuje linearni trend in spreminjanje temperature za 
        vsak mesec posebej.

\end{itemize}

Očitno je model B širši od modela A.

\subsection{a)}

Preizkusite model A znotraj modela B.

\subsection{b)}
Pri modeliranju je nevarno privzeti preširok model: lahko bi recimo postavili
model, po katerem je temperatura vsak mesec drugačna, neidvisno od ostalih
mesecev, a tak model bi bil neuporaben za napovedovanje. Akaikejeva 
informacija nam pomaga poiskati optimalni model – izberemo tistega, za katerega
je le-ta najmanjša. Akaikejeva informacija je sicer definirana z verjetjem, 
a pri linearni regresiji in Gaussovem modelu je le-ta ekvivalentna naslednji 
modifikaciji:

\[
    \text{AIC} := 2m + n \ln \text{RSS,}
    \]

kjer je m število parametrov, n pa je število opažanj. Kateri od zgornjih dveh
modelov ima manjšo Akaikejevo informacijo?


%%%%%%%%%%%%%%%%%%%%%%%%%%%%%%%%%%%%%%%%%%%%%%%%%%%%%%%%%%%%%%%%%%%%%%%%%%

\section{Literatura}

\end{document}
