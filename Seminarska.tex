\documentclass[12pt, a4paper]{article}
\usepackage[slovene,english]{babel}
\usepackage[utf8]{inputenc}
\usepackage{amsmath, amsfonts, amsthm, url, amssymb}
\usepackage{lmodern}
\usepackage{graphicx}
\usepackage[shortlabels]{enumitem}
\setlength{\parindent}{0cm}

\newtheorem{definicija*}{Definicija}
\newtheorem{trditev}[definicija*]{Trditev}
\newtheorem{posledica}[definicija*]{Posledica}
\newtheorem{izrek}[definicija*]{Izrek}
\newtheorem*{primer}{Primer}


\title{Seminarska naloga iz Statistike}
\author{Dejan Perić \\~ \\ }


\date{\today}

%%%%%%%%%%%%%%%%%%%%%%%%%%%%%%%%%%%%%%%%%%%%%%%%%%%%%%%%%%%%%%%%%%%%%%%%%%


\begin{document}
\selectlanguage{slovene}

\maketitle



%%%%%%%%%%%%%%%%%%%%%%%%%%%%%%%%%%%%%%%%%%%%%%%%%%%%%%%%%%%%%%%%%%%%%%%%%%

\section{Prva naloga}

V datoteki Kibergrad se nahajajo informacije o 43.886 družinah, 
ki stanujejo v mestu Kibergrad. Za vsako družino so zabeleženi 
naslednji podatki (ne boste potrebovali vseh):

\begin{itemize}
    \item Tip družine (od 1 do 3)
    \item Število članov družine
    \item Število otrok v družini
    \item Skupni dohodek družine
    \item Mestna četrt, v kateri stanuje družina (od 1 do 4)
    \item Stopnja izobrazbe vodje gospodinjstva (od 31 do 46)
\end{itemize}
    
\subsection{a)}
Vzemite enostavni slučajni vzorec 200 družin in na njegovi podlagi ocenite
povprečno število otrok na družino v Kibergradu.


\subsection{b)}
Ocenite standardno napako in postavite 95\% interval zaupanja.

\subsection{c)}
Vzorčno povprečje in ocenjeno standardno napako primerjajte s populacijskim
povprečjem in pravo standardno napako. Ali interval zaupanja iz prejšnje točke
pokrije populacijsko povprečje?

\subsection{d)}
Vzemite še 99 enostavnih slučajnih vzorcev in prav tako za vsakega določite
95\% interval zaupanja. Narišite intervale zaupanja, ki pripadajo tem 100 
vzorcem. Koliko jih pokrije populacijsko povprečje?

\subsection{e)}
Izračunajte standardni odklon vzorčnih povprečij za 100 prej dobljenih 
vzorcev. Primerjajte s pravo standardno napako za vzorec velikosti 200.

\subsection{f)}
Izvedite prejšnji dve točki še na 100 vzorcih po 800 družin. Primerjajte 
in razložite razlike s teorijo vzorčenja.


%%%%%%%%%%%%%%%%%%%%%%%%%%%%%%%%%%%%%%%%%%%%%%%%%%%%%%%%%%%%%%%%%%%%%%%%%%

\section{Druga naloga}

\subsection{a)}
Ocenite povprečje in standardni odklon za telesno temperaturo posebej pri
moških in posebej pri ženskah.

\subsection{b)}
Za povprečji iz prejšnje točke določite 95\% intervala zaupanja.

\subsection{c)}
Preizkusite domnevo, da imajo moški in ženske v povprečju enako telesno 
temperaturo.


%%%%%%%%%%%%%%%%%%%%%%%%%%%%%%%%%%%%%%%%%%%%%%%%%%%%%%%%%%%%%%%%%%%%%%%%%%

\section{Tretja naloga}

V datoteki Temp LJ se nahajajo izmerjene mesečne temperature v Ljubljani v 
letih od 1986 do 2020. Postavimo naslednja dva modela spreminjanja temperature
s časom: 

\begin{itemize}
     
    \item Model A: vključuje linearni trend in sinusno nihanje s periodo eno 
        leto.
    \item Model B: vključuje linearni trend in spreminjanje temperature za 
        vsak mesec posebej.

\end{itemize}

Očitno je model B širši od modela A.

\subsection{a)}

Preizkusite model A znotraj modela B.

\subsection{b)}
Pri modeliranju je nevarno privzeti preširok model: lahko bi recimo postavili
model, po katerem je temperatura vsak mesec drugačna, neidvisno od ostalih
mesecev, a tak model bi bil neuporaben za napovedovanje. Akaikejeva 
informacija nam pomaga poiskati optimalni model – izberemo tistega, za katerega
je le-ta najmanjša. Akaikejeva informacija je sicer definirana z verjetjem, 
a pri linearni regresiji in Gaussovem modelu je le-ta ekvivalentna naslednji 
modifikaciji:

\[
    \text{AIC} := 2m + n \ln \text{RSS,}
    \]

kjer je m število parametrov, n pa je število opažanj. Kateri od zgornjih dveh
modelov ima manjšo Akaikejevo informacijo?


%%%%%%%%%%%%%%%%%%%%%%%%%%%%%%%%%%%%%%%%%%%%%%%%%%%%%%%%%%%%%%%%%%%%%%%%%%

\section{Literatura}

\end{document}















